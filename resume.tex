%%%%%%%%%%%%%%%%%
% This is an example CV created using altacv.cls (v1.6.2, 28 Aug 2021) written by
% LianTze Lim (liantze@gmail.com), based on the
% Cv created by BusinessInsider at http://www.businessinsider.my/a-sample-resume-for-marissa-mayer-2016-7/?r=US&IR=T
%
%% It may be distributed and/or modified under the
%% conditions of the LaTeX Project Public License, either version 1.3
%% of this license or (at your option) any later version.
%% The latest version of this license is in
%%    http://www.latex-project.org/lppl.txt
%% and version 1.3 or later is part of all distributions of LaTeX
%% version 2003/12/01 or later.
%%%%%%%%%%%%%%%%

%% Use the "normalphoto" option if you want a normal photo instead of cropped to a circle
% \documentclass[10pt,a4paper,normalphoto]{altacv}

\documentclass[10pt,a4paper,ragged2e,withhyper]{assets/altacv}

%% AltaCV uses the fontawesome5 package.
%% See http://texdoc.net/pkg/fontawesome5 for full list of symbols.

% Change the page layout if you need to
\geometry{left=1.25cm,right=1.25cm,top=1.5cm,bottom=1.5cm,columnsep=1.2cm}

% The paracol package lets you typeset columns of text in parallel
\usepackage{paracol}


% Change the font if you want to, depending on whether
% you're using pdflatex or xelatex/lualatex
\ifxetexorluatex
  % If using xelatex or lualatex:
  \setmainfont{Lato}
\else
  % If using pdflatex:
  \usepackage[default]{lato}
\fi

% Change the colours if you want to
\definecolor{VividBlue}{HTML}{1058DF}
\definecolor{SlateGrey}{HTML}{2E2E2E}
\definecolor{LightGrey}{HTML}{666666}
% \colorlet{name}{black}
\colorlet{tagline}{VividBlue}
\colorlet{heading}{VividBlue}
\colorlet{headingrule}{VividBlue}
% \colorlet{subheading}{PastelRed}
\colorlet{accent}{VividBlue}
\colorlet{emphasis}{SlateGrey}
\colorlet{body}{LightGrey}

% Change some fonts, if necessary
% \renewcommand{\namefont}{\Huge\rmfamily\bfseries}
% \renewcommand{\personalinfofont}{\footnotesize}
% \renewcommand{\cvsectionfont}{\LARGE\rmfamily\bfseries}
% \renewcommand{\cvsubsectionfont}{\large\bfseries}

% Change the bullets for itemize and rating marker
% for \cvskill if you want to
\renewcommand{\itemmarker}{{\small\textbullet}}
\renewcommand{\ratingmarker}{\faCircle}

%% Use (and optionally edit if necessary) this .cfg if you
%% want to use an author-year reference style like APA(6)
%% for your publication list
% When using APA6 if you need more author names to be listed
% because you're e.g. the 12th author, add apamaxprtauth=12
\usepackage[backend=biber,style=apa6,sorting=ydnt]{biblatex}
\defbibheading{pubtype}{\cvsubsection{#1}}
\renewcommand{\bibsetup}{\vspace*{-\baselineskip}}
\AtEveryBibitem{\makebox[\bibhang][l]{\itemmarker}}
\setlength{\bibitemsep}{0.25\baselineskip}
\setlength{\bibhang}{1.25em}


%% Use (and optionally edit if necessary) this .cfg if you
%% want an originally numerical reference style like IEEE
%% for your publication list
% \usepackage[backend=biber,style=ieee,sorting=ydnt]{biblatex}
%% For removing numbering entirely when using a numeric style
\setlength{\bibhang}{1.25em}
\DeclareFieldFormat{labelnumberwidth}{\makebox[\bibhang][l]{\itemmarker}}
\setlength{\biblabelsep}{0pt}
\defbibheading{pubtype}{\cvsubsection{#1}}
\renewcommand{\bibsetup}{\vspace*{-\baselineskip}}


%% sample.bib contains your publications
% \addbibresource{sample.bib}

\begin{document}
\name{Pedro Ribeiro}
\tagline{Software Engineer Graduate}
% Cropped to square from https://en.wikipedia.org/wiki/Marissa_Mayer#/media/File:Marissa_Mayer_May_2014_(cropped).jpg, CC-BY 2.0
%% You can add multiple photos on the left or right
\photoR{2.5cm}{assets/pedro}
% \photoL{2cm}{Yacht_High,Suitcase_High}
\personalinfo{%
  % Not all of these are required!
  % You can add your own with \printinfo{symbol}{detail}
  \email{pedrouminhoribeiro@gmail.com}
  \phone{(+351) 916 669 693}
  % \mailaddress{Rua Santa Leocádia, porta 101, Semelhe}
  \location{Braga, PORTUGAL}
  % \homepage{marissamayr.tumblr.com}
  % \twitter{@marissamayer}
  % \linkedin{marissamayer}
  \github{pedroagribeiro} % I'm just making this up though.
%   \orcid{0000-0000-0000-0000} % Obviously making this up too.
  %% You can add your own arbitrary detail with
  %% \printinfo{symbol}{detail}[optional hyperlink prefix]
  % \printinfo{\faPaw}{Hey ho!}
  %% Or you can declare your own field with
  %% \NewInfoFiled{fieldname}{symbol}[optional hyperlink prefix] and use it:
  \NewInfoField{gitlab}{\faGitlab}[https://gitlab.com/]
  \gitlab{pedroagribeiro}
	%%
  %% For services and platforms like Mastodon where there isn't a
  %% straightforward relation between the user ID/nickname and the hyperlink,
  %% you can use \printinfo directly e.g.
  % \printinfo{\faMastodon}{@username@instace}[https://instance.url/@username]
  %% But if you absolutely want to create new dedicated info fields for
  %% such platforms, then use \NewInfoField* with a star:
  % \NewInfoField*{mastodon}{\faMastodon}
  %% then you can use \mastodon, with TWO arguments where the 2nd argument is
  %% the full hyperlink.
  % \mastodon{@username@instance}{https://instance.url/@username}
}

\makecvheader

%% Depending on your tastes, you may want to make fonts of itemize environments slightly smaller
\AtBeginEnvironment{itemize}{\small}

%% Set the left/right column width ratio to 6:4.
\columnratio{0.6}

% Start a 2-column paracol. Both the left and right columns will automatically
% break across pages if things get too long.
\begin{paracol}{2}

\cvsection{Education}

\cvevent{M.Sc. Computer Science}{University of Minho}{September 2020 -- Ongoing}{Braga, PORTUGAL}
\begin{itemize}
\item Focused in Application Engineering and Distributed Systems.
\item Took an Analysis and Testing of Software and a Software Deployment and Benchmarking course.
\end{itemize}

\divider

\cvevent{B.E Computer Science}{University of Minho}{September 2017 -- July 2020}{Braga, PORTUGAL}

% \divider

% \cvevent{Product Engineer}{Google}{23 June 1999 -- 2001}{Palo Alto, CA}

% \begin{itemize}
% \item Joined the company as employe \#20 and female employee \#1
% \item Developed targeted advertisement in order to use user's search queries and show them related ads
% \end{itemize}

\cvsection{Experience}
\cvevent{Internship @ AlticeLabs}{M.Sc. Dissertation}{October 2021 -- July 2021}{Aveiro, PORTUGAL} 
\begin{itemize}
    \item Description of the developed project.
\end{itemize}

\cvsection{Languages}

\cvskill{Portuguese}{5}
\divider

\cvskill{English}{4}
\divider

\cvsection{Techinal Skills}
\cvtag{HTML \& CSS}
\cvtag{REST API}
\cvtag{Java}
\cvtag{Python}
\cvtag{JavaScript}\\
\cvtag{MongoDB}
\cvtag{PostgreSQL}
\cvtag{Docker}
\cvtag{Kubernetes}
\cvtag{Ansible}
\cvtag{UML}
\cvtag{Git}

\cvsection{Personal Skills}
\cvtag{Honesty}
\cvtag{Persistance}
\cvtag{Dependability}
\cvtag{Responsability}

\cvsection{Other activities \& interests}
\cvevent{Graphic Design}{}{}{}
\begin{itemize}
    \item Produced graphic content for Sporting Clube de Cabreiros social media pages.
    \item Tools: \textbf{Adobe Photoshop, Adobe Illustrator, DaVinci Resolve}
\end{itemize}

%% Switch to the right column. This will now automatically move to the second
%% page if the content is too long.
\switchcolumn

\cvsection{Projects}

\cvevent{Beverage Managment}{University of Minho}{January 2021 -- June 2021}{}
\begin{itemize}
    \item Developed in the context of Application Engineering masters profile.
    \item The application aims to solve the middleman between high quantity buyers and the company that sells the drink through the establishment of a direct communication channel between the two parties.
    \item Technologies: \textbf{VueJS, Vuetify, JavaScript, Java, SpringBoot, MongoDB} 
\end{itemize}

\divider

\cvevent{Leonardo - Automated Tests Generation}{University of Minho}{January 2021 -- June 2021}{}
\begin{itemize}
    \item This project is part of a bigger project: Leonardo which aims to develop a platform that provides support to the studying process.
    \item A automated way to generate tests given a questions database was developed as well as an interface for the students to perform the tests.
    \item Technologies: \textbf{VueJS, JavaScript, Python, Flask, MongoDB}
\end{itemize}

\divider

\cvevent{Java Project Analyser}{University of Minho}{October 2020 -- January 2021}{}
\begin{itemize}
    \item Java projects from other students were analysed in a completely automated way with the goal of assessing their code quality. Tools were used in order to determine which the were the more common code smells and mispractices.
    \item Technologies: \textbf{Java, SonarQube, EvoSuite, Python, Bash}
\end{itemize}

\divider

\cvevent{Delegatewise - Task Sharing Simplified}{University of Minho}{February 2020 -- June 2020}{}
\begin{itemize}
    \item This application materializes a way of sharing tasks within a collaborating environment in a fair and unworried way, it automatically distributes the tasks by the registered users on a weekly basis.
    \item Technologies: \textbf{ReactJS, Gatsby, Spring Boot, MongoDB}
\end{itemize}

% \cvachievement{\faTrophy}{Courage I had}{to take a sinking ship and try to make it float}

% \divider

% \cvachievement{\faHeartbeat}{Persistence \& Loyalty}{I showed despite the hard moments and my willingness to stay with Yahoo after the acquisition}

% \divider

% \cvachievement{\faChartLine}{Google's Growth}{from a hundred thousand searches per day to over a billion}

% \divider

% \cvachievement{\faFemale}{Inspiring women in tech}{Youngest CEO on Fortune's list of 50 most powerful women}



\end{paracol}

\end{document}
